\chapter{Errores comunes}

La instalación de esta aplicación es muy sencilla, siempre y cuando tu sistema se encuentre actualizado como se menciona en el primer capítulo, aunque esto no siempre es así.

\section{Sistema no actualizado}

Si el sistema no se encuentra actualizado, la App Store lo indicará con un mensaje emergente diciendo que debes actualizar tu sistema a la versión más reciente para realizar la instalación de Xcode.\\

Sin embargo esto también puede presentarse cuando estas trabajando con una version de tu mac o macbook menos reciente. \href{https://stackoverflow.com/}{Stack Overflow} tiene un foro donde se ha realizado la corrección de este problema indicando cual versión de Xcode se puede instalar de acuerdo al sistema operativo con el que cuentas.

Para realizar la operación es necesario dirigirse a la siguiente \href{https://stackoverflow.com/questions/10335747/how-to-download-xcode-dmg-or-xip-file/10335943#10335943}{liga} para buscar aquella versión que corresponda con tu sistema.

\begin{figure}[hbtp!]
	\begin{center}	
		\fbox{\includegraphics[width=.5\textwidth]{images/stack}}
		\label{fig:stack}
		\caption{Página web donde se puede descargar la versión de Xcode compatible con tu sistema.}
	\end{center}
\end{figure}