\chapter{Instalación paso a paso}

Para poder realizar la instalación de la aplicación hay que seguir los siguientes pasos:

\begin{enumerate}
	\item Dirigirse a la App Store.
		\begin{figure}[hbtp!]
	\begin{center}	
			\fbox{\includegraphics[width=.5\textwidth]{images/appstore}}
			\label{fig:appstore}
			\caption{Pantalla de inicio de la App Store}
		\end{center}
	\end{figure}
		
	\item Se desplegará la pantalla de inicio de la app store como se puede ver en la figura \ref{fig:appstore}.

	\item Dirigirse al buscador que tiene integrada la aplicación como se ve en la figura \ref{fig:buscador}.	
	\begin{figure}[hbtp!]
		\begin{center}	
			\fbox{\includegraphics[width=.5\textwidth]{images/buscador}}
			\label{fig:buscador}
			\caption{Buscador de la aplicación.}
		\end{center}
	\end{figure}
	
	\break
	
	\item Presiona el botón \textbf{Enter} de tu teclado y se desplegará la búsqueda de las aplicaciones que tengan el nombre que escribiste.

\begin{figure}[hbtd!]
	\begin{center}	
		\fbox{\includegraphics[width=.5\textwidth]{images/busqueda}}
		\label{fig:busqueda}
		\caption{Resultados de la búsqueda de la palabra XCODE}
	\end{center}
\end{figure}

	\item Si la aplicación se encuentra instalada en tu sistema, aparecerá el icono con la leyenda \textbf{Open} para abrirla. Si esto no es así la leyenda dirá \textbf{GET}.
	
	\item El sistema solicitará tus permisos para realizar la instalación y esta concluirá de forma rápida y sencilla.
	
	\item Puedes abrir la aplicación y se debe mostrar un menú como se muestra en la siguiente figura \ref{fig:menu}:
\begin{figure}[hbtp!]
	\begin{center}	
		\fbox{\includegraphics[width=.5\textwidth]{images/menu}}
		\label{fig:menu}
		\caption{Menú principal de XCODE}
	\end{center}
\end{figure}

\end{enumerate}